% !TEX encoding = UTF-8 Unicode
\documentclass[a4paper]{article}

\usepackage{color}
\usepackage{url}
\usepackage[utf8]{inputenc} % make weird characters work
\usepackage{graphicx}
%\usepackage[nottoc]{tocbibind}

\usepackage[serbian, english]{babel}

\usepackage[unicode]{hyperref}
\hypersetup{colorlinks,citecolor=green,filecolor=green,linkcolor=blue,urlcolor=blue}

% used for code
\usepackage{color}
\usepackage{listings}
\usepackage{setspace}

\definecolor{Background}{rgb}{0.97,0.97,0.97}

\renewcommand{\lstlistingname}{Code}
\lstdefinelanguage{Python3}{
	language=Python,
	firstnumber=1,
	stepnumber=1,
	numbers=left,
	numbersep=5pt, 
	tabsize=4,
	basicstyle=\small\ttfamily,	
	morekeywords={import,from,class,def,for,while,if,is,in,elif,else,not,and,or,print,break,
		continue,return,True,False,None,access,as,,del,except,exec,finally,global,import,lambda,pass,print,raise,try,assert},
	stringstyle=\color{mymauve},
	commentstyle=\color{mygreen},  
	keywordstyle=\color{blue},
	backgroundcolor=\color{Background},	
	captionpos=b,  
	frame=single,	  
 	breakatwhitespace=false,
 	breaklines=true,
 	escapeinside={\%*}{*)},
 	extendedchars=true,
 	keepspaces=true,
	numberstyle=\tiny\color{mygray},
  	rulecolor=\color{black},
  	showspaces=false,
  	showstringspaces=false,
  	showtabs=false,
}

% this will be used when we want to show results
\lstdefinelanguage{Print}{
	numbers=none,
	tabsize=4,
	basicstyle=\small\ttfamily,
	frame=none	
}

\definecolor{mygreen}{rgb}{0,0.6,0}
\definecolor{mygray}{rgb}{0.5,0.5,0.5}
\definecolor{mymauve}{rgb}{0.58,0,0.82}

\lstset{ 
                  % show the filename of files included with \lstinputlisting; also try caption instead of title
}

\begin{document}

\title{Mining NBA}
\author{David Gavrilović}
% \date{9.~april 2015.}
\maketitle
\thispagestyle{empty}

\newpage

\tableofcontents
\thispagestyle{empty}

\newpage

\section{Stats 101}
\label{Stats_101}

\subsection{Traditional stats}
\label{Traditional_stats}


\begin{itemize}
	\item \textbf{Pos} - Position. Traditionaly, position can be one of the following: PG - point guard, SG - shooting guard, SF - small forward, PF - power forward and C - center. Nowdays, one player usually plays multiple positions and usually is oen of the: Point - primarly PG, Combo guard - plays PG and SG, Wing - SF and SG, Forward - PF and SF, Big - usually C but can also be PF.
	\item \textbf{G} - Games. Number of games player played in during a season.
	\item \textbf{GS} - Game started. Number of games player started. Cannot be greater than G.
	\item \textbf{MP} - Minutes played (Per game or total in a season).
	\item \textbf{FG} - Field goals.
	\item \textbf{FGA} - Field goals attempts.
	\item \textbf{FG\%} - Field goal percentage. Calculated as FG / FGA.
	\item \textbf{3P} - 3-point field goals.
	\item \textbf{3PA} - 3-point field goal attempts.
	\item \textbf{3P\%} - 3-point percentage. Calculated as 3P / 3PA.
	\item \textbf{2P} - 2-point field goals. 
	\item \textbf{2PA} - 2-point field goal attempts.
	\item \textbf{2P\%} - 2-point percentage. Calculated as 2P / 2PA.
	\item \textbf{FT} - Free throws.
	\item \textbf{FTA} - Free throw attempts.
	\item \textbf{FT\%} - Free throws percentage. Calculated as FT / FTA.
	\item \textbf{eFG\%} - Field goal percentage that takes into account that a 3-point field goal is, by one point, worth more than a 2-point field goal. Calculated as (FG + 0.5 * 3P) / FGA.
	\item \textbf{ORB} - Offensive rebounds.
	\item \textbf{TRB} - Defensive rebounds.
	\item \textbf{AST} - Assists.
	\item \textbf{STL} - Steals.
	\item \textbf{BLK} - Blocks.
	\item \textbf{TOV} - Turnovers.
	\item \textbf{PF} - Personal fouls.
	\item \textbf{PTS or PPG} - Points or Points per game.
	
\subsection{Advanced stats}
\label{Advanced_stats}

...

\end{itemize}

\pagebreak

\addcontentsline{toc}{section}{References}
\appendix
\bibliography{ref}
\bibliographystyle{unsrt}
\appendix

\end{document}
